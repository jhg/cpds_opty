\documentclass[a4paper, 11pt]{article}
\usepackage[utf8]{inputenc} % Change according your file encoding
\usepackage{graphicx}
\usepackage{url}

%opening
\title{Seminar Report: Opty}
\author{Maria Gabriela Valdes and Victoria Beleuta}
\date{\today{}}

\begin{document}

\maketitle

\section{Introduction}

During this lab, we implemented a transaction server using optimistic concurrency control in Erlang . Opty is a concurrency control method applied to transactional systems such as relational database management systems. It assumes that multiple transactions can complete without interfering with each other, while using data resources without acquiring locks. Before committing, each transaction checks if any other transaction has changed the entry. If it hasn't been changed the transation can commit, if there are conflicts, the transaction rollsback and restarts.
 
\section{Work done}

We have completed the code files provided to correctly implement the algorithm. Our source code can be found in the src folder. To start the algorithm locally you have to call the function \textit{start} of module \textit{opty} with six arguments in the following order: number of clients, number of entries, number of updates, number of reads, duration of experiments and size of the subset of entries to be used by each client. For example: opty:start(3,4,1,1,5,2). With this command the Opty algorithm will start with three clients, and a store with four entries, one write operation, one read operation, with a duration of 5 seconds and a subset of 2 entries out of the total provided. \\
To start the algorithm with the server and clients in different Erlang nodes you first have to start two different Erlang environments. For example:
\begin{itemize}
\item In one terminal type erl -sname optyserver. This will represent an Erlang node name optyserver@server.\\
\item In another different terminal type erl -sname clients. This will represent a Erlang node name clients@server.\\
\end{itemize}\\
%
First we start the server in the node optyserver@server with the command \textit{opty:start\_server(10)}, where 10 is the number of entries. To start the clients you have to call, in the clients node, the function \textit{start\_clients} of module \textit{opty} with seven arguments: number of clients, number of entries, number of updates, number of reads, duration of experiments, the name of the server’s node and the size of the subset of entries. For example: 
\textit{opty:start\_clients(5, 10, 1, 1, 5, optyserver@server, 3)}.\\

\section{Experiments}

\textbf{In the same machine:}\\\\
\textbf{i)} different number of concurrent clients in the system;\\
We ran the following commands to see how the algorithm runs:\\
\begin{itemize}
\item opty:start(1,10,1,1,5,3)\\
\includegraphics[scale=0.5]{images/exp-i-1.png} \\
\item opty:start(5,10,1,1,5,3)\\
\includegraphics[scale=0.5]{images/exp-i-2.png} \\
\item opty:start(10,10,1,1,5,3)\\
\includegraphics[scale=0.5]{images/exp-i-3.png} \\
\item opty:start(20,10,1,1,5,3)\\
\includegraphics[scale=0.5]{images/exp-i-4.png} \\
\item opty:start(40,10,1,1,5,3)\\
\includegraphics[scale=0.5]{images/exp-i-5.png} \\
\item opty:start(80,10,1,1,5,3)\\
\includegraphics[scale=0.5]{images/exp-i-6a.png} \\
\includegraphics[scale=0.5]{images/exp-i-6b.png} \\
\end{itemize}
%
\textbf{ii)} different number of entries in the store;\\
We ran the following commands to see how the algorithm runs:\\
\begin{itemize}
\item opty:start(5,1,1,1,1,1)\\
\includegraphics[scale=0.5]{images/exp-ii-1.png} \\
\item opty:start(5,3,1,1,1,3)\\
\includegraphics[scale=0.5]{images/exp-ii-2.png} \\
\item opty:start(5,5,1,1,1,5)\\
\includegraphics[scale=0.5]{images/exp-ii-3.png} \\
\item opty:start(5,10,1,1,1,10)\\
\includegraphics[scale=0.5]{images/exp-ii-4.png} \\
\item opty:start(5,100,1,1,1,100)\\
\includegraphics[scale=0.5]{images/exp-ii-5.png} \\
\item opty:start(5,1000,1,1,1,1000)\\
\includegraphics[scale=0.5]{images/exp-ii-6.png} \\
\end{itemize}
%
\textbf{iii)} different number of write operations per transaction;\\
We ran the following commands to see how the algorithm runs:\\
\begin{itemize}
\item opty:start(5,10,1,1,5,3)\\
\includegraphics[scale=0.5]{images/exp-iii-1.png} \\
\item opty:start(5,10,5,1,5,3)\\
\includegraphics[scale=0.5]{images/exp-iii-2.png} \\
\item opty:start(5,10,10,1,5,3)\\
\includegraphics[scale=0.5]{images/exp-iii-3.png} \\
\item opty:start(5,10,20,1,5,3)\\
\includegraphics[scale=0.5]{images/exp-iii-4.png} \\
\item opty:start(5,10,40,1,5,3)\\
\includegraphics[scale=0.5]{images/exp-iii-5.png} \\
\item opty:start(5,10,80,1,5,3)\\
\includegraphics[scale=0.5]{images/exp-iii-6.png} \\
\end{itemize}
%
\textbf{iv)} different ratio of read and write operations per transaction;\\
We ran the following commands to see how the algorithm runs:\\
\begin{itemize}
\item opty:start(5,10,2,4,1,5)\\
\includegraphics[scale=0.5]{images/exp-iv-1.png} \\
\item opty:start(5,10,4,2,1,5)\\
\includegraphics[scale=0.5]{images/exp-iv-2.png} \\
\item opty:start(5,10,3,9,1,5)\\
\includegraphics[scale=0.5]{images/exp-iv-3.png} \\
\item opty:start(5,10,9,3,1,5)\\
\includegraphics[scale=0.5]{images/exp-iv-4.png} \\
\item opty:start(5,10,1,10,1,5)\\
\includegraphics[scale=0.5]{images/exp-iv-5.png} \\
\item opty:start(5,10,10,1,1,5)\\
\includegraphics[scale=0.5]{images/exp-iv-6.png} \\
\item opty:start(5,10,1,100,1,5)\\
\includegraphics[scale=0.5]{images/exp-iv-7.png} \\
\item opty:start(5,10,100,1,1,5)\\
\includegraphics[scale=0.5]{images/exp-iv-8.png} \\
\end{itemize}
%
\textbf{v)} different percentage of accessed entries with respect to the total number of entries;\\
We ran the following commands to see how the algorithm runs:\\
\begin{itemize}
\item opty:start(5,10,2,2,1,2)\\
\includegraphics[scale=0.5]{images/exp-v-1.png} \\
\item opty:start(5,10,2,2,1,3)\\
\includegraphics[scale=0.5]{images/exp-v-2.png} \\
\item opty:start(5,10,2,2,1,4)\\
\includegraphics[scale=0.5]{images/exp-v-3.png} \\
\item opty:start(5,10,2,2,1,5)\\
\includegraphics[scale=0.5]{images/exp-v-4.png} \\
\item opty:start(5,10,2,2,1,8)\\
\includegraphics[scale=0.5]{images/exp-v-5.png} \\
\end{itemize}

\textbf{In different machines:}\\\\


\section{Open questions}

\textbf{1)} What is the impact of each of these parameters on the success rate? Is the success rate the same for the different clients?\\
\textbf{2)} If we run this in a distributed Erlang network, where is the handler running?\\


\section{Personal opinion}



\end{document}
